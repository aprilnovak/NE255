\documentclass{beamer}

\mode<presentation>
{
  \usetheme{Madrid}
  \usecolortheme{default}
  \usefonttheme{serif}
  \setbeamertemplate{navigation symbols}{}
  \setbeamertemplate{caption}[numbered]
}

\usepackage[english]{babel}
\usepackage[utf8x]{inputenc}
\usepackage[version=3]{mhchem}

\usepackage{pgfpages}
\usepackage{xcolor}
\usepackage{esvect}
\usepackage{graphicx}
\usepackage[compatibility=false]{caption}
\usepackage{subcaption}
\usepackage{soul}


\pgfpagesuselayout{resize to}[%
  physical paper width=8in, physical paper height=6in]

% for circled numbers
\usepackage{tikz}
\newcommand*\circled[1]{\tikz[baseline=(char.base)]{
            \node[shape=circle,draw,inner sep=2pt] (char) {#1};}}

\title[PRONGHORN]{Streamline-Upwind Petrov-Galerkin Stabilization in PRONGHORN}
\author{April Novak}
\institute{UC Berkeley}
\date{\today}

\begin{document}

\begin{frame}
  \titlepage
\end{frame}

\begin{frame}{Outline}
\begin{enumerate}
\item What is PRONGHORN?
	\begin{itemize}
		\item Why MOOSE?
		\item Why PRONGHORN?
	\end{itemize}
\item The Navier-Stokes equations
\item Why do we need to stabilize these equations?
\item What are our stabilization options?
\item SUPG stabilization
\end{enumerate}
\end{frame}

\section{What is PRONGHORN?}

\begin{frame}{What is PRONGHORN?}

\begin{figure}[H]
\centering
\includegraphics[width=0.6\linewidth]{/Users/aprilnovak/projects/wolverine/doc/manual/figures/PronghornAnimal.pdf}
\end{figure}

\begin{itemize}
  \item Porous-media Computational Fluid Dynamics (CFD) finite element code
  \item Developed on the MOOSE framework
\end{itemize}

\end{frame}

%%%%%

\begin{frame}{Why MOOSE (Multiphysics Object-Oriented Simulation Environment)?}

\begin{itemize}
  \item Develop \textit{modern} nuclear software
  \item Using a robust numerical method (finite elements)
  \item Using the State-of-the-Art PETSc nonlinear solver
  \item For easy multiphysics simulations of advanced reactors
\end{itemize}

\end{frame}

%%%%%

\section{Why PRONGHORN?}

\begin{frame}{Why porous media?}

\begin{figure}[H]
\centering
\includegraphics[width=0.6\linewidth]{/Users/aprilnovak/projects/wolverine/doc/manual/figures/TMSR_closeup.pdf}
\caption{Close-up view of pebble mesh in a small test reactor.}
\label{fig:TMSRmesh}
\end{figure}

\begin{enumerate}
\item The mesh resolution needed makes the simulation \textit{very} expensive (\(\approx10^{10}\) elements)
\item Imprecise knowledge of the physical structure
\end{enumerate}

\end{frame}

%%%%%

\begin{frame}{Why porous media?}

\begin{itemize}
\item Instead of high-fidelity computation, just compute averaged quantities!
\item Provides an intermediate point between very detailed CFD and 1-D systems-level codes such as RELAP-7\newline
\end{itemize}

\begin{figure}[H]
\centering
\begin{subfigure}{.25\textwidth}
  \centering
  \includegraphics[width=1.0\linewidth]{/Users/aprilnovak/projects/wolverine/doc/manual/figures/NonPorousMesh.pdf}
  \caption{Non-porous media model}
\end{subfigure}
\begin{subfigure}{.25\textwidth}
  \centering
  \includegraphics[width=1.0\linewidth]{/Users/aprilnovak/projects/wolverine/doc/manual/figures/PorousMesh.pdf}
  \caption{Porous media model}
\end{subfigure}
\caption{Difference between a mesh for (a) generic CFD calculations (no boundary layer mesh shown) and (b) a porous media model.}
\end{figure}

\end{frame}

%%%%%

\begin{frame}{The Navier-Stokes Equations}

\begin{itemize}
\item PRONGHORN solves the porous-media form of the compressible, inviscid Navier-Stokes equations (a.k.a. the Euler equations)
\end{itemize}

\end{frame}

%%%%%

\begin{frame}{The Navier-Stokes Equations}
\small
\begin{equation}
\begin{aligned}
\frac{\partial\rho}{\partial t} + \nabla\cdot(\rho\vv{V})=&\ 0\\
\rho\frac{\partial\vv{V}}{\partial t}+\nabla\cdot(\rho\vv{V}\otimes\vv{V})-\rho \vv{g}+\nabla P-\nabla\cdot\bar{\bar{\tau}}=&\ 0\\
\frac{\partial (\rho E)}{\partial t}+\nabla\cdot(\rho h\vv{V})-\rho \vv{g}\cdot\vv{V}-\nabla\cdot\left(\bar{\bar{\tau}}\vv{V}-k\nabla T\right)+\alpha(T-T_\infty)-q=&\ 0\\
\end{aligned}
\end{equation}

\begin{equation}
\bar{\bar{\tau}}= \mu\left(\nabla \vv{V}+(\nabla \vv{V})^T\right)-\frac{2\mu}{3}\nabla\cdot\vv{V}\textbf{I}
\end{equation}

\normalsize
\begin{tabular}{l l l l}
\(\rho\) & density & \(\vv{V}\) & velocity\\
\(\vv{g}\) & gravity acceleration & \(P\) & pressure\\
\(\mu\) & viscosity & \(\textbf{I}\) & identity matrix\\
\(E\) & total energy & \(h\) & enthalpy\\
\(k\) & thermal conductivity & \(T\) & temperature\\
\(\alpha\) & heat transfer coefficient & \(q\) & volumetric heat source\\
\end{tabular}

\end{frame}

%%%%%

\begin{frame}{What makes the Navier-Stokes equations difficult to solve?}

\begin{itemize}
\item \textit{Highly} nonlinear due to convection terms
	\begin{itemize}
		\item You're not guaranteed a solution
		\item No longer meet the ``Best Approximation Theorem'' - numerical results can be way off!
	\end{itemize}
\item 6 tightly-coupled equations
	\begin{itemize}
		\item What order do you solve these in?
	\end{itemize}
\item Unlike neutronics, it is actually better to neglect diffusive (viscous) effects
	\begin{itemize}
		\item Don't have to resolve boundary layers (no-slip condition)
		\item Easier boundary conditions
		\item \textbf{But then our equations become hyperbolic :(}
	\end{itemize}
\end{itemize}

\end{frame}

%%%%%

\begin{frame}{PRONGHORN Governing Equations}
\small
\begin{equation}
\begin{aligned}
\epsilon\frac{\partial\rho_f}{\partial t} + \nabla\cdot(\epsilon\rho_f\vv{V})=&\ 0\\
\epsilon\frac{\partial(\rho_f\vv{V})}{\partial t}+\nabla\cdot(\epsilon\rho_f\vv{V}\otimes\vv{V}+\epsilon P\textbf{I})-\epsilon\rho_f\vv{g}+\color{red}\left(D\mu\epsilon+F\epsilon^2\rho_f|\vv{V}|\right)\vv{V}\color{black}=&\ 0\\
\epsilon\frac{\partial (\rho_f E)}{\partial t}+\nabla\cdot(\epsilon h_f\rho_f\vv{V})-\epsilon\rho_f \vv{g}\cdot\vv{V}+\alpha(T_f-T_s)-q_f=&\ 0\\
1-\epsilon)\rho_sC_{p,s}\frac{\partial T_s}{\partial t}-\nabla\cdot(\kappa_s\nabla T_s)+\alpha(T_s-T_f)-q_s=&\ 0\\
\end{aligned}
\end{equation}
\normalsize
\begin{itemize}
\item Neglect viscous forces
\item \color{red}Extra terms\ \color{black} are added to account for frictional and form drag (think of spoilers on cars)
\end{itemize}

\end{frame}

%%%%%

\section{SUPG Stabilization}

\begin{frame}{Stabilization}

\begin{itemize}
\item All fluids codes that contain convective terms require some form of stabilization
\end{itemize}

\begin{figure}[H]
  \centering
  \includegraphics[width=5.5cm]{/Users/aprilnovak/projects/wolverine/doc/manual/figures/1D_convection_diffusion.pdf}
  \caption{Solutions to the 1-D advection-diffusion problem for various \(Pe\).}
\end{figure}

\small
\begin{equation}
Pe=\frac{\nabla\cdot(\vv{V}T)}{k_f\nabla^2 T}= \frac{Vh}{2k}
\end{equation}
\normalsize
\end{frame}

%%%%%

\begin{frame}{Why are convective equations difficult to solve?}

\begin{itemize}
\item Take the 1-D convection-diffusion equation as a model problem:
\end{itemize}

\begin{equation}
V\frac{d\phi}{dx}-k\frac{d^2\phi}{dx^2}=0
\end{equation}

\begin{itemize}
\item If we assemble the global stiffness matrix and load vector, using linear Lagrange elements with the Bubnov-Galerkin method, we obtain:
\end{itemize}

\begin{equation}
\begin{bmatrix} a & b & 0 & 0 & 0 & 0\\
			c & d + a & b & 0 & 0 & 0\\
			0 & c & d + a & b & 0 & 0\\
			0 & 0 & c & d + a & b & 0\\
			 & & & &  & \cdots \\\end{bmatrix}
\begin{bmatrix}\cdots \\ \phi_{n-1}\\\phi_{n}\\\phi_{n+1}\\\cdots\end{bmatrix}=
\begin{bmatrix}0 \\ 0 \\ 0 \\ 0 \\ 0\end{bmatrix}
\end{equation}

where \(a=-Pe_{el}+1, b=Pe_{el}-1, c=-Pe_{el}-1, d=Pe_{el}+1\). 

\end{frame}

%%%%%

\begin{frame}
\begin{itemize}
\item Then, an equation for a typical node is:

\begin{equation}
\label{eq:1D_nodal_eq}
(-Pe_{el}-1)\phi_{a-1}+2\phi_a+(Pe_{el}-1)\phi_{a+1}=0
\end{equation}

\item Assume a solution \(\phi=r^j\). This gives you two possible solutions:

\begin{equation}
\phi_j=1^j\quad\quad\text{or}\quad\quad\color{red}\phi_j=\left(-\frac{Pe_{el}+1}{Pe_{el}-1}\right)^j\color{black}
\end{equation}

\begin{equation}
Pe_{el}\equiv\frac{Vh}{2k}=\frac{\text{velocity}\cdot\text{mesh spacing}}{2\cdot\text{diffusivity}}
\end{equation}

\end{itemize}
\end{frame}

\begin{frame}{Without Stabilization...}

\begin{figure}[H]
  \centering
  \includegraphics[width=7.5cm]{/Users/aprilnovak/projects/wolverine/doc/manual/figures/1D_convection_diffusion.pdf}
  \caption{Solutions to the 1-D advection-diffusion problem for various \(Pe\).}
\end{figure}

\end{frame}

%%%%%

\begin{frame}

\begin{figure}[H]
  \centering
  \includegraphics[width=7.5cm]{/Users/aprilnovak/projects/wolverine/doc/manual/figures/continuity_eqn_no_stabilization.pdf}
  \caption{Solution to the continuity equation with no stabilization - the result should be \(\rho=\sin{(2\pi x)}\).}
\end{figure}

\end{frame}

%%%%%

\begin{frame}{Stabilization Options}

\begin{itemize}
\item \textbf{The problem}: We don't want to excessively refine the mesh when we only care about engineering-scale results
\item \textbf{The early-on solution(s)}
	\begin{itemize}
		\item Simply add some diffusion!
		\item Modify the quadrature rule
		\item Modify the shape functions
	\end{itemize}
\end{itemize}

\begin{figure}[H]
  \centering
  \includegraphics[width=5.5cm]{/Users/aprilnovak/projects/wolverine/doc/manual/figures/1DShapeFunctions.pdf}
  \caption{Petrov-Galerkin upwinding shape functions for a 1-D element.}
\end{figure}

\end{frame}

%%%%%

\begin{frame}{Streamline Upwind Petrov-Galerkin (SUPG) Stabilization}

\begin{itemize}
\item Earlier methods failed to realize how to extend what worked for specific cases to general, 3-D problems
\item Hughes and Brooks (1982) developed a \textit{consistent} stabilization method (SUPG)
	\begin{itemize}
		\item Generalizes to 3-D
		\item Consistent
		\item No over-diffusion - the solution is preserved
	\end{itemize}
\item The realization that a Petrov-Galerkin formulation could give good stabilizing results is one of the main reasons why anyone even tries to use finite elements for fluids applications
\end{itemize}

\end{frame}

%%%%%
\begin{frame}{Streamline Upwind Petrov-Galerkin (SUPG) Stabilization}

\begin{itemize}
\item Add a component to every shape function \(\psi\) that is proportional to the amount of convection:

\begin{equation}
\tilde{\psi}=\psi+\tau\vv{V}\cdot\nabla \psi
\end{equation}

\begin{equation}
\tau=\frac{h_e}{2\|\vv{V}\|_2}
\end{equation}

\item The overall weak form becomes:

\begin{equation}
\int_{\Omega}R(u)\psi d\Omega+\sum_{elem}\int_{\Omega}R(u)\tau\vv{V}\cdot\nabla \psi d\Omega=0
\end{equation}

where \(R(u)\) is the residual for the finite element solution \(u\).
\end{itemize}
\end{frame}

\begin{frame}{Streamline Upwind Petrov-Galerkin (SUPG) Stabilization}
\begin{itemize}
\item Why does this work?
\end{itemize}

\begin{equation}
\tilde{\psi}=\psi+\tau\vv{V}\cdot\color{red}\nabla \psi\color{black}
\end{equation}

The weak form for the \(\nabla^2 u\) (diffusion) kernel is:

\begin{equation}
\int_{\Omega}^{}\color{green}\nabla u\color{black}\cdot\color{red}\nabla\psi\color{black} d\Omega
\end{equation}

The weak form for the \(\vv{V}\cdot\nabla u\) (advective) kernel is:

\begin{equation}
\int_{\Omega}^{}\vv{V}\cdot\nabla u\psi d\Omega\quad\rightarrow\quad\int_{\Omega}^{}\vv{V}\cdot\color{green}\nabla u\color{black}\left(\psi+\tau\vv{V}\cdot\color{red}\nabla \psi\color{black}\right) d\Omega
\end{equation}

\end{frame}

%%%%%

\begin{frame}{SUPG Stabilization: Results}

\begin{itemize}
\item I implemented SUPG stabilization in PRONGHORN
	\begin{itemize}
		\item Obtained nodally-exact results
		\item Performed verification testing of all new kernels added
		\item Reduced runtime by allowing smaller \(\Delta t\)
	\end{itemize}
\end{itemize}

\end{frame}

%%%%%

\begin{frame}{Obtaining the correct results}

\begin{figure}[H]
  \centering
  \includegraphics[width=7.5cm]{/Users/aprilnovak/projects/wolverine/doc/manual/figures/1D_convection_diffusion_SUPG.pdf}
  \caption{Solutions to the 1-D advection-diffusion problem for various \(Pe\) with SUPG stabilization.}
\end{figure}

\end{frame}

%%%%%

\begin{frame}{Obtaining the correct results}

\begin{figure}[H]
\centering
\begin{subfigure}{.425\textwidth}
  \centering
  \includegraphics[width=1.0\linewidth]{/Users/aprilnovak/projects/wolverine/doc/manual/figures/NoStabilization.pdf}
  \caption{No stabilization}
\end{subfigure}
\begin{subfigure}{.425\textwidth}
  \centering
  \includegraphics[width=1.0\linewidth]{/Users/aprilnovak/projects/wolverine/doc/manual/figures/Stabilization.pdf}
  \caption{Stabilization}
\end{subfigure}
\caption{Stabilization effect on obtaining the correct solution of \(\rho=\sin{(2\pi x)}+5\). }
\end{figure}

\end{frame}

%%%%%

\begin{frame}{Theoretical (or better) convergnce}

\begin{table}[H]
\caption{Convergence rates for PRONGHORN SUPG kernels.}
\centering
\begin{tabular}{l l c c c c}
\hline\hline
Kernel  & Linear & Quadratic\\ [0.5ex]
\hline
\texttt{ContinuityEqnSUPG} 				& 2.11 - 3.02 & 3.00\\
\texttt{MomConvectiveFluxSUPG} 			& 2.11 - 3.02 & 3.00\\
\texttt{MomPressureGradientSUPG} 			& 2.11 - 3.02 & 3.00\\
\texttt{MomFrictionForceSUPG} 			& 3.00 & 2.93\\
\texttt{MomGravityForceSUPG } 			& 2.00 & 2.93\\
\texttt{FluidEnergyConvectiveFluxSUPG} 		& 2.08 & 2.97\\
\texttt{FluidEnergyGravityForceSUPG} 		& 2.10 & 2.97\\
\texttt{FluidSolidConvectionSUPG} 			& 2.00	     & 3.00\\			
\hline
\end{tabular}
\label{table:TestMatrixPlanSUPG}
\end{table}

\end{frame}

\begin{frame}{Reducing runtime}
\begin{itemize}
\item Stabilization gives you results 100-1000 times faster! (by allowing bigger \(\Delta t\))
\end{itemize}

\begin{figure}[H]
\centering
\begin{subfigure}{0.45\textwidth}
  \centering
  \includegraphics[width=1.0\linewidth]{/Users/aprilnovak/projects/wolverine/doc/manual/figures/Runtime_unstabilized_continuity_eqn_full_caseb.pdf}
  \caption{Unstabilized, continuity}
\end{subfigure}
\begin{subfigure}{0.45\textwidth}
  \centering
  \includegraphics[width=1.0\linewidth]{/Users/aprilnovak/projects/wolverine/doc/manual/figures/Runtime_stabilized_continuity_eqn_full_caseb.pdf}
  \caption{Stabilized, continuity}
\end{subfigure}
\caption{Difference in runtime between the (a) unstabilized and (b) stabilized continuity equations as a function of \(\Delta t\) for an end time of 60 seconds.}
\end{figure}

\end{frame}

\begin{frame}{Reducing runtime}
\begin{figure}[H]
  \centering
  \includegraphics[width=7cm]{/Users/aprilnovak/projects/wolverine/doc/manual/figures/runtime_ratio.pdf}
  \caption{Ratio between the runtime without and with stabilization as a function of velocity magnitude for an end time of 60 seconds.}
\end{figure}
\end{frame}

%%%%%

\end{document}
